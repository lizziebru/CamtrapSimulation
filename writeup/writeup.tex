\documentclass[11pt]{article}
\usepackage[utf8]{inputenc}
\usepackage{setspace}
\usepackage{lineno}
\usepackage{authblk}
\usepackage{natbib}
\usepackage[margin=2cm]{geometry}
\usepackage{tabularx}
\newcommand{\supersc}[1]{\ensuremath{^{\textrm{#1}}}}
\newcommand{\subsc}[1]{\ensuremath{_{\textrm{#1}}}}
\usepackage{booktabs}
\bibliographystyle{unsrtnat}
\usepackage{graphicx}



\title{\textbf{Estimating average speed of movement from camera trap data}}
\author[1]{Lizzie Bru}
\affil[1]{School of Life Sciences, Imperial College London, Silwood Park Campus, Ascot SL5 7PY, UK}
\date{}

\onehalfspacing
\linenumbers

\begin{document}
	
	\maketitle
	
	\newpage
	
	\section{Abstract}

	
	\section{Introduction}
	
	intro of diff methods for working out animal density using CTs:
	- see Rovero et al. 2013 for a nice summary
	- plus also Li et al. 2022
	
	introduce REM
	- first put forward in Rowcliffe et al. 2008
	- then improved in Rowcliffe et al. 2011 (distance detection probability paper)
	- then again in Rowcliffe et al. 2016
	- first field test by Rovero et al. 2009
	
	to do REM well, need to be able to work out animal average speed of movement
	
	current methods to do this:
	- outlined in Rowcliffe et al. 2016
	- but not happy with those methods -- estimates of speed are biased
	
	
	OUTLINE THE PROBLEM:
	- Statistical problem
	- Have observations of animal speeds
	- We want to use these for abundance estimation methods for monitoring biodiversity
	- So there are some statistical models that allow you to do that
	- Using gas model type thing
	- There are those biases in our speed estimations
	- Want to incorporate these
	- Biases include e.g. some animals e.g. herbivores show v bimodal movement
			→ so basically coming up with statistically sound ways of estimation to minimise that bias
	- This is based on existing data that is already manually digitised
	
	
	specific biases to investigate:
	
	- Speed proportional to trap rate
		- M has considered this but only as a linear relationship - might be too simplistic
	- V high speeds likely to be missed by cameras
		- M hasn’t given this much thought yet - so far thinks it’s negligible - but: would deserve some thought


	BIAS: long step-length = fast speed = more likely that CT will miss the final step (cause they’ll leave the detection zone) - so will miss the crucial big step length!
	
	
	
	categorising spp movements to make multiple types of simulation:
	- issue is there's not much in the literature about this
	- most stuff is on less fine-scale movement patterns - e.g. \cite{shaw2020causes}
	- even stuff that is supposedly on finer-scale movement patterns is limited by usually being telemetry data-based and therefore not that fine-scale and restricted to only certain individuals 
		- e.g. \cite{leblond2010drives}
	
	
	
	in intro and discussion: 
	- include how understanding biases in CT data/info is useful to help us better combine it with telemetry/accelerometry data to get the best info using multiple methods complementarily
	-- could maybe test this after the project using my stuff on Rory Wilson's data 
	
	--> additional potential collaboration with Rory Wilson:
	- he has a measure he developed - ODBA - to do with speed and tortuosity - from accelerometry data (they're coming up with much more real-life data using GPS and accelerometry data together)- v simple formula
	- measures cost of movement
	- could apply ODBA to simulation to think about costs of movement
	- useful to see if could use this model to give insights into the costs of movement and to see why these distributions look the way they do - i.e. could use the model to explain relationships in evo sense
	- e.g. what combos are most likely in nature? which are the CTs getting?
	--> keep as a side project - could maybe add in at the end as a nice biology bit
	
	
	fleshing out biases is important to warn ppl and make suggestions e.g. maybe better to combine CTs
	
	fine-scale analysis of speed and tortuosity = hard to come by so this stuff is super important
	
	
	
	
	Why we expect there to be a bias in oversampling of faster speeds:
	- animals should be getting caught in particular areas - bc of the correlation between speed and tortuosity: so they're moving round in circles in patches which might not be in the dz: so you're basically less likely to sample them
	--> therefore more likely to pick up faster speeds
	
	
	
	
	
	\section{Methods}
	
	
	REM equation:
	
		\begin{equation}
			D = \frac{y}{t} \cdot \frac{\pi}{Vr(2 + \theta)}
	\end{equation}
	
	Where \textit{D} is population density estimated using this method, \textit{y} is the number of independent photographic events, \textit{t} is total camera survey effort (i.e. trap rate), \textit{V} is average animal movement speed, and \textit{r} and \begin{math}\theta\end{math} are the radius and angle of the camera trap detection zone, respectively. 
	
	
	distance detection probability stuff:
	- just did this for radius bc recent data suggests that actually there's a pretty high detection probability around the edges of the dz so the detection probability with angle is pretty uniform --> ask Marcus for reference on this?
	- detection probability with angle is therefore informed by just hazard rate logistic mix from Rowcliffe et al. fitted to data I have (Regent's park + India)
	- the detection zone of a camera trap is entirely probabilistic though of course - there's no hard detection zone boundary so hence why it was important to incorporate this detection probability
	
	
	
	
	
	
	
	
	
	
	\section{Results}
	
	
	
	\newpage
	
	\section{Discussion}
	
	
	

	\section{Acknowledgements}

	
	\bibliography{writeup}
	
\end{document}