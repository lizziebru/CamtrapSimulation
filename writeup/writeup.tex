\documentclass[a4paper,12pt,twoside]{report}
\usepackage[utf8]{inputenc}
\usepackage{setspace}
\usepackage{lineno}
\usepackage{fancyhdr}
\usepackage{authblk}
\usepackage{natbib}
\usepackage[left=2cm,right=2cm,top=2cm,bottom=3cm]{geometry}
\usepackage{tabularx}
\newcommand{\supersc}[1]{\ensuremath{^{\textrm{#1}}}}
\newcommand{\subsc}[1]{\ensuremath{_{\textrm{#1}}}}
\usepackage{booktabs}
\bibliographystyle{unsrtnat}
\usepackage{graphicx}
\usepackage[textwidth=1.5cm, textsize=small]{todonotes} % Add option "disable" to disable todo notes
\newcommand{\tocite}{\todo[color=red!40]{cite}}


\title{\textbf{Estimating average speed of movement from camera trap data}}
\author[1]{Elisabeth Bru}
\affil[1]{School of Life Sciences, Imperial College London, Silwood Park Campus, Ascot SL5 7PY, UK}
\date{August 2022}

\onehalfspacing
\linenumbers

\begin{document}
	
	\pagestyle{fancy}
	\maketitle \todo{ask Francis for help on removing line numbers on title page}
	   \vfill 
	{\centering A thesis submitted in partial fulfilment of the requirements for the degree of Master of Research at Imperial College London\par}
	{\centering Submitted for the MRes in Computational Methods in Ecology and Evolution\par}
	
	\newpage
	
	\section{Abstract}

	
	\section{Introduction}
	
	Reliable estimation of the density of wild animal populations is essential in wildlife conservation, management, and ecology. Uses range widely from informing prioritisation of species and area protection, for instance via helping predict extinction risk \cite{purvis2000predicting}, to furthering our understanding of population-level processes, for instance relating to behaviour \cite{judge1997rhesus} or disease transmission \cite{lindgren2000impact}. Estimating animal density is difficult to achieve, however, particularly for species which are cryptic, not individually-recognisable, or live in inaccessible terrains.

	Many methods have been developed to estimate animal population density in a given area. These commonly involve collecting using visual observations or sensor technology, such as camera traps \cite{rowcliffe2008surveys} or acoustic detectors \cite{acevedo2006field}. Such data are then used to estimate density using a variety of methods including more traditional visually-based distance sampling surveys \cite{buckland2001introduction}, using track counts together with body mass-day range scaling \cite{keeping2014rapid}, using a cue-count approach \cite{marques2009estimating}, using camera trap photographic rate \cite{carbone2001use}, and using Random Encounter Modelling (REM) \cite{rowcliffe2008estimating, lucas2015generalised}. 

	The REM method \cite{rowcliffe2008estimating} estimates density from camera trap data by modelling the underlying observation process. The model linearly correlates density and camera trapping rate (number of photographs per unit time), giving density as a function of  trapping rate, average animal speed of movement, and the dimensions of the camera trap detection zone (angle and radius) \cite{rowcliffe2008estimating}. The REM method was first tested in an animal park enclosure with known density of species, where it provided accurate estimates in three out of four cases \cite{rowcliffe2008estimating}. The first field test of the REM method showed support for the linear relationship between density and trapping rate but highlighted inaccuracies of the density estimates calculated, prompting further work to refine the accuracy of the method \cite{rovero2009camera}. Improvements have been made to the REM method, including quantifying and including camera sensitivity \cite{marcus2011quantifying}, and devising a method to estimate average speed of movement by fitting size-biased probability distributions to speed observations from camera trap data \cite{rowcliffe2016wildlife}.

	Reliably estimating average speed of movement is essential to ensure the accuracy of the REM method \cite{rowcliffe2016wildlife}. Currently, the method to estimate average speed involves calculating the speeds of individual movement sequences (when an animal crosses the detection zone of a camera trap and is recorded on two or more frames) by dividing distance covered by sequence duration, then averaging out these speeds using maximum likelihood, fitting one of three size-biased distributions (log-normal, gamma, and Weibull) or the non-parametric equivalent: harmonic mean \cite{rowcliffe2016wildlife}. While empirical validation of this method showed that it yields acceptably accurate speed estimates, the authors highlight a need for improvement of the distributions to better account for biases in the sampling of animal speeds by camera traps \cite{rowcliffe2016wildlife}. There are two main biases which are likely to particularly affect the accuracy of speed estimation. Firstly, animals moving faster are more likely to encounter camera traps, suggesting that speed is proportional to trap rate. Fitting size-biased models in principle corrects for this. However, slight asymmetry in the empirical distributions of speeds on the log scale and discrepancies between how well different types of distributions correct for this bias suggest room for improvement and further exploration of how best to correct for this bias\cite{rowcliffe2016wildlife}. The second bias which is likely to affect speed estimation from camera trap data is that fast speeds are likely to be 'missed' by cameras. This stems from the limitations of fixed image capture rates in camera traps: when animals cross the camera detection zone fast enough, they may be only captured either in a single frame or in no frames at all. This means that the speed of that sequence of movement cannot be calculated, since two or more frames are required as a minimum to do so, and therefore is not taken into account in average speed estimation. Current methods do not address this bias. 
	
	In this study I therefore explore whether the current methods of estimating average travel speed are sufficiently accurate to be used in practice. I specifically investigate whether we see evidence of these two biases in the speeds picked up by camera traps, and whether currently-proposed models sufficiently correct for them if so. To do this, I model the movement of animals and the detection of sequences of this movement by camera traps using a simulation.  
	
	
	
	
	
	
	
	
	
	

	intro of diff methods for working out animal density using CTs:
	- see Rovero et al. 2013 for a nice summary
	- plus also Li et al. 2022 \tocite
	
	introduce REM
	- first put forward in Rowcliffe et al. 2008
	- then improved in Rowcliffe et al. 2011 (distance detection probability paper)
	- then again in Rowcliffe et al. 2016
	- first field test by Rovero et al. 2009\todo[color=yellow!40, inline]{help me}
	
	to do REM well, need to be able to work out animal average speed of movement
	
	current methods to do this:
	- outlined in Rowcliffe et al. 2016
	- but not happy with those methods -- estimates of speed are biased
	
	
	OUTLINE THE PROBLEM:
	- Statistical problem
	- Have observations of animal speeds
	- We want to use these for abundance estimation methods for monitoring biodiversity
	- So there are some statistical models that allow you to do that
	- Using gas model type thing
	- There are those biases in our speed estimations
	- Want to incorporate these
	- Biases include e.g. some animals e.g. herbivores show v bimodal movement
			→ so basically coming up with statistically sound ways of estimation to minimise that bias
	- This is based on existing data that is already manually digitised
	
	
	specific biases to investigate:
	
	- Speed proportional to trap rate
		- M has considered this but only as a linear relationship - might be too simplistic
	- V high speeds likely to be missed by cameras
		- M hasn’t given this much thought yet - so far thinks it’s negligible - but: would deserve some thought


	BIAS: long step-length = fast speed = more likely that CT will miss the final step (cause they’ll leave the detection zone) - so will miss the crucial big step length!
	
	
	
	categorising spp movements to make multiple types of simulation:
	- issue is there's not much in the literature about this
	- most stuff is on less fine-scale movement patterns - e.g. \cite{shaw2020causes}
	- even stuff that is supposedly on finer-scale movement patterns is limited by usually being telemetry data-based and therefore not that fine-scale and restricted to only certain individuals 
		- e.g. \cite{leblond2010drives}
	
	
	
	in intro and discussion: 
	- include how understanding biases in CT data/info is useful to help us better combine it with telemetry/accelerometry data to get the best info using multiple methods complementarily
	-- could maybe test this after the project using my stuff on Rory Wilson's data 
	
	--> additional potential collaboration with Rory Wilson:
	- he has a measure he developed - ODBA - to do with speed and tortuosity - from accelerometry data (they're coming up with much more real-life data using GPS and accelerometry data together)- v simple formula
	- measures cost of movement
	- could apply ODBA to simulation to think about costs of movement
	- useful to see if could use this model to give insights into the costs of movement and to see why these distributions look the way they do - i.e. could use the model to explain relationships in evo sense
	- e.g. what combos are most likely in nature? which are the CTs getting?
	--> keep as a side project - could maybe add in at the end as a nice biology bit
	
	
	fleshing out biases is important to warn ppl and make suggestions e.g. maybe better to combine CTs
	
	fine-scale analysis of speed and tortuosity = hard to come by so this stuff is super important
	
	
	
	
	Why we expect there to be a bias in oversampling of faster speeds:
	- animals should be getting caught in particular areas - bc of the correlation between speed and tortuosity: so they're moving round in circles in patches which might not be in the dz: so you're basically less likely to sample them
	--> therefore more likely to pick up faster speeds
	
	
	
	nice intro-y bit stolen from Palencia et al. 2019 (estimating day range..):
	- traditionally, movement ecology and abundance of organisms have been studied separately --> but we should actually be studying them together bc they're two parts of the same problem 
	
	
	
	
% TODAY'S WORK

	
	
	
	
\section{Methods}

\subsection{Data used}

To inform decisions on simulating animal movement and the camera trap detection process, I used pre-processed camera trap data from Regent's Park, Panama, and Northeast India. 

\subsection{Simulation}

The structure of the simulation is as follows:
\medskip

\noindent1. Generate an animal movement path with known input average movement speed
\medskip

\noindent2. Overlay this path with a camera trap which captures parts of the path which pass through the camera field of view as position data
\medskip

\noindent3. Calculate the speed of each movement sequence (section of the path which crosses the camera detection zone)
\medskip

\noindent4. Average these sequence speeds using methods described in \cite{rowcliffe2016wildlife} to estimate average movement speed
\medskip

\noindent I wrote all code in RStudio version 3.6.3 (2020-02-29) and ran the animal movement path generation using the Imperial College London High Performance Computing service.

\subsubsection{Animal movement path}

I simulate an animal movement path as a sequence of x and y coordinates generated using a correlated random walk, confined to a toroidal arena of 40 x 40 metres. I define the movement path such that animals take one step per second, so speed is varied via the distance covered during each step. I also define speeds to be correlated with turn angle at each step.
\medskip

\noindent The variable input parameters I provide are:
\medskip

\noindent1. Mean log speed
\medskip

\noindent Animal speeds tend to be log-normally distributed, so I input the log of the desired average speed. I used speeds in the range 0.02m/s - 1.00m/s, informed by both the data and common understanding of how fast animals can move. 
\medskip

\noindent2. Log standard deviation of mean log speed
\medskip

\noindent I generate an input log speed standard deviation appropriately scaled to the input log speed using a coefficient of variation between mean log speed and mean log speed standard deviation calculated from the data:

\begin{equation}
	CV = \frac{log(sd)}{log(\mu)}
\end{equation}

\noindent where \textit{CV} is the coefficient of variation between mean log speed and mean log speed standard deviation, \textit{sd} is the standard deviation of average speed, and \begin{math}\mu\end{math} is average speed. I calculated one \textit{CV} for each body size category (small and large) by fitting a normal distribution to the logged speeds of individual species and dividing the standard deviation by the mean of the distribution. I then calculated a mean \textit{CV} for small and large body size categories by averaging out individual species' \textit{CV}s.
\medskip

\noindent For this study, I provided the following fixed values for the remaining input parameters:
\medskip

\noindent1. Animal body size
\medskip

\noindent I used two categories: small and large, defined by the size threshold of 4kg according to methods used in \cite{marcus2011quantifying}. At this stage of the simulation, this parameter only affects the log standard deviation of mean log speed. I fixed this as small for this study. 
\medskip

\noindent2. Autocorrelation in speed (from 0 (no autocorrelation) to 1 (total autocorrelation)): 0.9
\medskip

\noindent3. Number of steps taken in the path: 5e5
\medskip

\noindent4. Probability of turning at each step (from 0 (never turn) to 1 (always turn)): 0.5
\medskip

\noindent5. Mean vonMises concentration parameter (\begin{math}\kappa\end{math}) for turn angle at each step (analogous to standard deviation): 2
\medskip



% END OF TODAY'S WORK




	
	
	
	
	
	
	
	
	
	
	
	
	
	
	
	
	
	\subsubsection{Parameters: camera trap}
	
	detection zone size and detection probability
	- dz is probabilistic
	- detection probability methods - used Rowcliffe et al. 2011 fitting the models to the data available
	
	
	image capture rate
	
	
	\subsubsection{Parameters: animal movement path}
	
	speedSD
	- did some runs to investigate which is best
	- maybe include some plots to show this?
	
	pTurn
	- ditto
	
	speedCor
	- ditto
	
	kTurn
	
	mean speed
	
	no. of steps
	- this one should just be tuned so that can get enough data 
	- chose 5e5
	
	spatial limits of the path
	- need it to be big enough for us to see our biases
	
	variation in turnangle
	
	whether to correlate turnangle with speed
	
	\subsection{Using the simulation to investigate biases}
	
	what runs I did
	
	what I was plotting/testing for to see these biases
	
	
	
	
	\subsection{Investigating factors affecting biases}
	
	
	
	
	\subsection{Modelling to improve speed estimation}
	

	
	
	REM equation:
	
		\begin{equation}
			D = \frac{y}{t} \cdot \frac{\pi}{Vr(2 + \theta)}
	\end{equation}
	
	Where \textit{D} is population density estimated using this method, \textit{y} is the number of independent photographic events, \textit{t} is total camera survey effort (i.e. trap rate), \textit{V} is average animal movement speed, and \textit{r} and \begin{math}\theta\end{math} are the radius and angle of the camera trap detection zone, respectively. 
	
	
	distance detection probability stuff:
	- just did this for radius bc recent data suggests that actually there's a pretty high detection probability around the edges of the dz so the detection probability with angle is pretty uniform --> ask Marcus for reference on this?
	- detection probability with angle is therefore informed by just hazard rate logistic mix from Rowcliffe et al. fitted to data I have (Regent's park + India)
	- the detection zone of a camera trap is entirely probabilistic though of course - there's no hard detection zone boundary so hence why it was important to incorporate this detection probability
	
	
	
	
	\section{Results}
	
	
	
	\newpage
	
	\section{Discussion}
	
	
	something that the simulation couldn't address - how the reaction of animals to cameras could be a cause of bias in speeds picked up -- see discussion in Rowcliffe et al. 2008 for this -- could be worth bringing up bc they also recommend futher work on this -- and it could interact with these biases that we've been investigating 
	
	
	

	\section{Acknowledgements}

	
	\bibliography{writeup}
	
\end{document}